\chapter{Conclusiones}\label{Chapter5}

El capítulo final muestra los resultados alcanzados con el trabajo e informa las
mejoras necesarias a futuro.

\section{Resultados obtenidos}

% El funcionamiento del dispositivo es el esperado.
% La cantidad de recursos utilizados es la esperada. 
% Es mejor usar software abierto.
% La implementación de test mediante python, pytest y cocotb facilito mucho el desarrollo.
% El CI ayudo a encontrar problema que surgian en algún móídulo mientras se desarrollava otro.

% Hacer un desarrollo basado en una norma supuso un desafío extra

% En el desarrollo del trabajo se aplican los conocimientos adquiridos durante la
% cursada de la carrera de especialización, en particular las asignaturas de gestión
% de proyectos, Programación de microcontroladores, diseño de circuitos impresos,
% diseño para la manufacturabilidad e ingeniería de software en sistemas embebi-
% dos. El aprendizaje adquirido sobre la gestión de problemas, estructuración de
% tareas y prevención de retrasos permite finalizar la carrera con una visión com-
% pleta sobre un sistema embebido.

En el desarrollo de este trabajo, que consistió en el diseño de un módulo SDI
de triple tasa de datos para dispositivos lógicos programables, se han
alcanzado satisfactoriamente los objetivos planteados al inicio del trabajo
considerando que que solo uno de los supuestos iniciales se cumplió. El
dispositivo diseñado funciona según lo esperado, cumpliendo con las
especificaciones establecidas en la fase de diseño. Este éxito demuestra la
viabilidad del enfoque adoptado y de la funcionalidad del módulo desarrollado.

Una de las conclusiones clave de este trabajo es que la cantidad de recursos
de FPGA utilizados se alineó estrechamente con las proyecciones iniciales.
Este resultado subraya la precisión en la planificación y la eficiencia del
diseño. Además, el uso de herramientas de código abierto a lo largo de la gran
mayoría del trabajo no solo demostró ser viable, sino también preferible.
Proporcionó una mayor flexibilidad y control sobre el proceso de desarrollo,
permitiendo ajustes y modificaciones de manera ágil y eficiente, lo cual fue
crucial para el éxito del trabajo.

El proceso de implementación de pruebas, utilizando herramientas como Python,
Pytest y Cocotb, simplificó significativamente el desarrollo. Esta estrategia
facilitó una verificación eficaz y eficiente del diseño, permitiendo detectar
y corregir errores de manera temprana. La adopción de prácticas de integración
continua tuvo un rol fundamental en el mantenimiento de la integridad del
proyecto. Permitió identificar problemas que surgían en algunos módulos mientras
se desarrollaban otros, asegurando así que el proyecto avanzara de manera
cohesiva y consistente.

Abordar el diseño basándose en una normativa específica presentó desafíos
adicionales, pero también estableció un marco de referencia claro que guió el
desarrollo. Este enfoque normativo no solo aseguró la compatibilidad y el
cumplimiento del módulo SDI desarrollado, sino que también proporcionó una
estructura organizativa que facilitó el manejo de la complejidad del proyecto.
Además, la documentación de dispositivos similares fue de gran utilidad a la
resolver dudas de implementación y estructurar el flujo de datos.

Durante el desarrollo de este trabajo, se aplicaron de manera integral los
conocimientos adquiridos a lo largo de la carrera de especialización. Las
habilidades y conceptos aprendidos en temáticas como gestión de proyectos,
programación de dispositivos lógicos programables, definición de
requerimientos, especificaciones y pruebas para verificación y validación y
diseño para la reutilización y mutabilidad fueron fundamentales para el éxito
del proyecto. Esta aplicación práctica de la teoría a un proyecto complejo y
desafiante no solo reforzó estos conocimientos, sino que también proporcionó
una experiencia enriquecedora.

Finalmente, este trabajo no solo cumplió con la gran mayoría de los objetivos
técnicos propuestos, sino que también sirvió como una valiosa experiencia de
aprendizaje. Demostró la importancia de combinar la teoría con la práctica en
el campo de la ingeniería y subrayó el valor de un enfoque metodológico y
reflexivo en el desarrollo de soluciones tecnológicas avanzadas. Este proyecto
prepara el terreno para futuras investigaciones y desarrollos.

\section{Trabajo futuro}

% Llevar el modulo hasta los estandares de 12G.
% Terminar de testear los formatos restantes.
% Agregar formal verification a los módulos.
% Implementar los test funcionales en UVM para que el testeo sea estandar.
% Probar en los equipos de la empresa VideoSwitch.
% Hacer mediciones con equipamiento específico para los estandares y con capacidad de análizar la integridad de la señal de TV.

Mirando hacia el futuro, existen varias áreas en las que el proyecto podría
expandirse y mejorar. Uno de los objetivos principales sería llevar el módulo
hasta los estándares de 12G, lo que ampliaría significativamente su aplicabilidad
y permitiría su uso en una gama más amplia de aplicaciones de alta velocidad.
Además, sería crucial terminar de probar los formatos restantes para asegurar
una compatibilidad y funcionalidad completas del módulo en diferentes escenarios
de uso.

La adición de verificación formal a los módulos representaría un paso adelante
en el aseguramiento de la calidad y la confiabilidad del diseño. Esta
metodología puede ayudar a identificar problemas que no son fácilmente
detectables mediante pruebas convencionales. Implementar las pruebas funcionales
en UVM (del inglés, \textit{Universal Verification Methodology}) estándarizaría el
proceso de verificación, facilitando la reutilización de los \textit{test} y
mejorando la eficiencia del proceso de verificación. Se plantea el uso de PyUVM,
que se integra perfectamente con Cocotb, presenta todas las funcionalidades de
UVM en SystemVerilog, pero al estar basado en Python agiliza el desarrollo.

Además, sería valioso probar el módulo en equipos específicos de la empresa
VideoSwitch, lo que proporcionaría una validación final del diseño en un entorno
de producción real. Por último, realizar mediciones con equipamiento específico
para los estándares y con capacidad de analizar la integridad de la señal de
televisión sería esencial para garantizar que el módulo cumple con los requisitos
de calidad y rendimiento más exigentes.
